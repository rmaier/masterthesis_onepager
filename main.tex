% ****** Start of file apssamp.tex ******
%
%   This file is part of the APS files in the REVTeX 4.1 distribution.
%   Version 4.1 of REVTeX, October 2009
%
%   Copyright (c) 2009 The American Physical Society.
%
%   See the REVTeX 4 README file for restrictions and more information.
%
% TeX'ing this file requires that you have AMS-LaTeX 2.0 installed
% as well as the rest of the prerequisites for REVTeX 4.1
%
% See the REVTeX 4 README file
% It also requires running BibTeX. The commands are as follows:
%
%  1)  latex apssamp.tex
%  2)  bibtex apssamp
%  3)  latex apssamp.tex
%  4)  latex apssamp.tex
%
\documentclass[%
a4paper,
% reprint,
%superscriptaddress,
%groupedaddress,
%unsortedaddress,
%runinaddress,
%frontmatterverbose, 
% preprint,
%showpacs,preprintnumbers,
% nofootinbib,
% nobibnotes,
% bibnotes,
% amsmath,amssymb,
aps,
% prl,
pra,
% prb,
%rmp,
%prstab,
%prstper,
 longbibliography,
%floatfix,
 lengthcheck,%
% linenumbers,
]{revtex4-1}

\usepackage{graphicx}% Include figure files
\usepackage{dcolumn}% Align table columns on decimal point
\usepackage{bm}% bold math
% \usepackage{hyperref}% add hypertext capabilities
\usepackage{upgreek}
% \usepackage[utf8]{inputenc}
%\usepackage[mathlines]{lineno}% Enable numbering of text and display math
%\linenumbers\relax % Commence numbering lines
\usepackage[utf8]{inputenc}
%\usepackage[showframe,%Uncomment any one of the following lines to test 
%%scale=0.7, marginratio={1:1, 2:3}, ignoreall,% default settings
%%text={7in,10in},centering,
%%margin=1.5in,
%%total={6.5in,8.75in}, top=1.2in, left=0.9in, includefoot,
%%height=10in,a5paper,hmargin={3cm,0.8in},
%]{geometry}
% \bibliographystyle{}
\begin{document}

%\preprint{APS/123-QED}

\title{Entwicklung eines Anlaufmodells für das Lean Startup -- Exposé}
% Zukünftige Forschungsfelder für die Entwicklung des Anlaufmanagements im Mittelstand -- Exposé}
%\thanks{A footnote to the article title}%

\author{Rudolph Ribeiro Maier}
% \email{rm1988@mailbox.tu-berlin.de}
\email{rudolph.m.ribeiromaier@campus.tu-berlin.de}
\affiliation{Fachgebiet Qualitätswissenschaft,
Institut für Werkzeugmaschinen und Fabrikbetrieb, TU Berlin}


\date{\today}% It is always \today, today,
             %  but any date may be explicitly specified

\begin{abstract}
\textbf{Motivation \& Problemstellung:}
Die produzierende Industrie findet sich heutzutage in einem zunehmend dynamischen Wettbewerbsumfeld wieder, welches vielschichtige Herausforderungen mit sich bringt \cite{Renner2012}. Die hauptsächlichen Herausforderungen liegen in steigenden Innovationsgeschwindigkeiten, kürzeren Produktlebenszyklen und einer höheren Variantenvielfalt \cite{Kuhn2002,Stauder2016}. Um dem durch die Globalisierung verstärkten Wettbewerb standzuhalten, müssen produzierende Unternehmen innovative Produkte und Dienstleistungen anbieten und sich zunehmend kundenorientiert aufstellen \cite{Surbier2014}. 
Eine zentrale Rolle wird hier dem Anlauf von Serienprodukten zugeschrieben. Aufgrund immer kürzer werdender Produktlebenszyklen rücken Kosten und Zeitaufwand in den Vordergrund \cite{Winkler2007}. So hat der Anlauf einen signifikanten Einfluss auf den wirtschaftlichen Erfolg des Produkts und die Time-to-Volume \cite{Klocke16}. Selbst ein um wenige Monate verschobener Verkaufsstart kann über Erfolg oder Misserfolg des Produkts entscheidend sein \cite{Schuh2008a}. Die Bedeutung der Serienanläufe findet bisher in der Wissenschaft keine angemessene Aufarbeitung \cite{Dyckhoff2012}. 
\end{abstract}

                             % Classification Scheme.
%\keywords{Suggested keywords}%Use showkeys class option if keyword
                              %display desired
\maketitle

% \tableofcontents

\section{Fokus der Arbeit}

Der Trend zur Konzentration auf Kernkompetenzen sorgt dafür, dass in großen Unternehmen immer mehr Wertschöpfungsanteile an Zulieferer abgegeben werden  \cite{Hilmola2015, Wildemann2008}. Der Gesamtanlauf setzt sich fortan aus vielen lokalen Einzelanläufen zusammen \cite{Zimolong2006}. Im Zuge dessen erhöht sich die Komplexität des Gesamtanlaufs. An dieser Stelle setzt das Konzept des Lean Start-up (LSU) an. Das Lean Start-up ist eine Businessmethode für dynamische Unternehmen oder Projekte, die hohen Risiken und Unsicherheiten ausgesetzt sind. %TODO cite!! possible sources: Edison2015a/2, Eisenmann2013/1 . -> search notes at home about 'Unternehmen und Projekte'
Im Jahre 2008 führte RIES zum ersten Mal den Begriff Lean Start-up ein \cite{Ries2008}. 
Das Konzept des Anlaufmanagements befasst sich mit der Planung, Durchführung und Steuerung des Serienanlaufs \cite[S.8]{Kuhn2002}. Hauptziele sind die Beherrschung und die zeitliche Verkürzung der Anlaufphase \cite{Kuhn2002, Schmitt2015}.

Bei der Analyse der Literatur zu LSU lässt sich feststellen, dass der Themenkomplex Anlaufmanagement bisher noch nicht abgebildet wird. Da jedoch eine Beherrschung reibungsloser Serienanläufe ein entscheidender Wettbewerbsvorteil ist, sind hier erhebliche Potentiale für das LSU zu erwarten \cite[S.XI]{Bischoff2007}. Darauf basierend lässt sich folgende Hypothese für die Arbeit ableiten: 

\textbf{Hypothese}: Der Themenkomplex Anlaufmanagement findet in der Businessmethode Lean Start-up keine angemessene Beachtung. Die Beherrschung eines reibungslosen Serienanlaufs ist jedoch ein erheblicher Wettbewerbsvorteil. 

Basierend auf der Hypothese werden folgende Forschungsfragen aufgestellt, die in der Abschlussarbeit beantwortet werden müssen: 

\textbf{FF 1}: Wie kann der Serienanlauf im LSU gestaltet werden? 

\textbf{FF 1.1}: Was zeichnet das LSU mit Hinblick auf das Anlaufmanagement aus? Welche Anforderungen werden gestellt?

\textbf{FF 1.2}: Welche Aspekte des Anlaufmanagements sind für das LSU von Bedeutung? 

\textbf{FF 1.3}: Wie könnte ein Anlaufmanagement-Ansatz für das LSU auf Basis des Stands der Wissenschaft zum Thema Anlaufmanagement aussehen?  


\section{Herangehensweise}
Die Abschlussarbeit wird eine Literaturarbeit. In der Einführung erfolgt eine knappe Darstellung der zu behandelnden Themen Lean Start-up und Anlaufmanagement. Im Hauptteil wird zunächst der Stand der Wissenschaft zum Thema Lean Start-up skizziert. Den größeren Teil bildet eine umfassende Literaturanalyse zum Stand der Wissenschaft des Anlaufmanagements. Die Literaturrecherche erfolgt nach fest definierten Kriterien. Für die Literaturanalyse werden mit Hilfe des Tools \textit{Atlas.ti} alle relevanten Textstellen gecoded, d.h. identifiziert und nachvollziehbar dokumentiert. Anhand der  Ergebnisse wird anhand von 15-20 Quellen der Stand der Wissenschaft dargestellt. Im nächsten Abschnitt werden für das Lean Start-up nicht berücksichtigte Anforderungen an das Anlaufmanagement ermittelt und daraus eine Art Anlaufmodell abgeleitet. 

Die Validierung der Ergebnisse erfolgt durch Zitierung der Quellen. Auf eine Validierung durch Experten, Fragebögen oder empirische Untersuchungen wird aufgrund des großen Umfangs verzichtet. 

\vspace{1cm}

\bibliography{../50_latex/main} % Produces the bibliography via BibTeX.

\end{document}
%
% ****** End of file apssamp.tex ******
