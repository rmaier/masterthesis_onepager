% ****** Start of file apssamp.tex ******
%
%   This file is part of the APS files in the REVTeX 4.1 distribution.
%   Version 4.1 of REVTeX, October 2009
%
%   Copyright (c) 2009 The American Physical Society.
%
%   See the REVTeX 4 README file for restrictions and more information.
%
% TeX'ing this file requires that you have AMS-LaTeX 2.0 installed
% as well as the rest of the prerequisites for REVTeX 4.1
%
% See the REVTeX 4 README file
% It also requires running BibTeX. The commands are as follows:
%
%  1)  latex apssamp.tex
%  2)  bibtex apssamp
%  3)  latex apssamp.tex
%  4)  latex apssamp.tex
%
\documentclass[%
a4paper,
% reprint,
%superscriptaddress,
%groupedaddress,
%unsortedaddress,
%runinaddress,
%frontmatterverbose, 
% preprint,
%showpacs,preprintnumbers,
% nofootinbib,
% nobibnotes,
% bibnotes,
% amsmath,amssymb,
aps,
% prl,
pra,
% prb,
%rmp,
%prstab,
%prstper,
 longbibliography,
%floatfix,
 lengthcheck,%
% linenumbers,
]{revtex4-1}

\usepackage{graphicx}% Include figure files
\usepackage{dcolumn}% Align table columns on decimal point
\usepackage{bm}% bold math
% \usepackage{hyperref}% add hypertext capabilities
\usepackage{upgreek}
% \usepackage[utf8]{inputenc}
%\usepackage[mathlines]{lineno}% Enable numbering of text and display math
%\linenumbers\relax % Commence numbering lines
\usepackage[utf8]{inputenc}
%\usepackage[showframe,%Uncomment any one of the following lines to test 
%%scale=0.7, marginratio={1:1, 2:3}, ignoreall,% default settings
%%text={7in,10in},centering,
%%margin=1.5in,
%%total={6.5in,8.75in}, top=1.2in, left=0.9in, includefoot,
%%height=10in,a5paper,hmargin={3cm,0.8in},
%]{geometry}
% \bibliographystyle{}
\begin{document}

%\preprint{APS/123-QED}

\title{Entwicklung eines Anlaufmodells für das Lean Startup -- Exposé}
% Zukünftige Forschungsfelder für die Entwicklung des Anlaufmanagements im Mittelstand -- Exposé}
%\thanks{A footnote to the article title}%

\author{Rudolph Ribeiro Maier}
% \email{rm1988@mailbox.tu-berlin.de}
\email{rudolph.m.ribeiromaier@campus.tu-berlin.de}
\affiliation{Fachgebiet Qualitätswissenschaft,
Institut für Werkzeugmaschinen und Fabrikbetrieb, TU Berlin}


\date{\today}% It is always \today, today,
             %  but any date may be explicitly specified

\begin{abstract}
\textbf{Motivation \& Problemstellung:}
Die produzierende Industrie findet sich heutzutage in einem zunehmend dynamischen Wettbewerbsumfeld wieder, welches vielschichtige Herausforderungen mit sich bringt \cite{Renner2012}. Die hauptsächlichen Herausforderungen liegen in steigenden Innovationsgeschwindigkeiten, kürzeren Produktlebenszyklen und einer höheren Variantenvielfalt \cite{Kuhn2002,Stauder2016}. Um dem durch die Globalisierung verstärkten Wettbewerb standzuhalten, müssen produzierende Unternehmen innovative Produkte und Dienstleistungen anbieten und sich zunehmend kundenorientiert aufstellen \cite{Surbier2014}. 
Eine zentrale Rolle wird hier dem Anlauf von Serienprodukten zugeschrieben. Aufgrund immer kürzer werdender Produktlebenszyklen rücken Kosten und Zeitaufwand in den Vordergrund \cite{Winkler2007}. So hat der Anlauf einen signifikanten Einfluss auf den wirtschaftlichen Erfolg des Produkts und die Time-to-Volume \cite{Klocke16}. Selbst ein um wenige Monate verschobener Verkaufsstart kann über Erfolg oder Misserfolg des Produkts entscheidend sein \cite{Schuh2008a}. Die Bedeutung der Serienanläufe findet bisher in der Wissenschaft keine angemessene Aufarbeitung \cite{Dyckhoff2012}. 
\end{abstract}

                             % Classification Scheme.
%\keywords{Suggested keywords}%Use showkeys class option if keyword
                              %display desired
\maketitle

% \tableofcontents

\section{Fokus der Arbeit}
Der Trend zur Konzentration auf Kernkompetenzen sorgt dafür, dass in großen Unternehmen immer mehr Wertschöpfungsanteile an Zulieferer abgegeben werden  \cite{Hilmola2015, Wildemann2008}. Der Gesamtanlauf setzt sich fortan aus vielen lokalen Einzelanläufen zusammen \cite{Zimolong2006}. Daraus resultieren höhere Abhängigkeiten zwischen größeren Unternehmen und den Zulieferern, die meist mittelständische Unternehmen sind. 

Die Abschlussarbeit soll sich im Speziellen mit dem Serienanlauf im KMU und SME als Zulieferer für größere Unternehmen beschäftigen, da hier erhebliches Verbesserungspotential erkennbar ist \cite[S.18]{Dombrowski2009a}. So gibt es in KMU meist keine Anlaufprozesse. Da es in KMU oft keine Stabsstellen gibt, werden Anläufe von den Mitarbeitern oft zusätzlich zum Tagesgeschäft gesteuert \cite{Dombrowski2009}. %TODO kein Zugriff auf Primärquelle D.Spath!! 
Mangelnde finanzielle und zeitliche Kapazitäten sowie fehlendes Know-how verhindern eine nachvollziehbare Dokumentation sowie proaktive Maßnahmen \cite{Zimolong2006,Dombrowski2009a}. 

Weiterhin soll untersucht werden, wie der Auftraggeber den Anlaufprozess des Lieferanten unterstützen kann. Größere Unternehmen verfügen in der Regel über mehr Ressourcen und teilweise eigene Anlaufprozesse. Im Zuge der Verlagerung der Wertschöpfungsanteile, gewinnt die Innovationskraft von Modul- und Systemlieferanten zunehmend an Bedeutung für den Erfolg eines Produktes \cite{Kuhn2002}. Ein erfolgreiches und effizientes Anlaufmanagement in KMU ist im Sinne der Entwicklung einer nachhaltigen Partnerschaft für Auftraggeber und Lieferant von großer Bedeutung. \textit{Wildemann} erkennt hier das Potenzial von Einspareffekten sowie Nutzung erheblicher Wettbewerbsvorteile auf beiden Seiten \cite{Wildemann2008}.

\textit{Dyckhoff} und \textit{Scholz} sind zu der Erkenntniss gekommen, dass das Thema weder in Industrie noch in der Wissenschaft hinreichend Beachtung findet \cite{Dyckhoff2012, Scholz2010}, weshalb hier keine zufriedenstellenden Ergebnisse zu erwarten sind.
Ziel der Arbeit ist, einen Überblick über den Stand der Forschung zu geben und einen Entwurf für ein Anlaufmodell zu entwickeln. 

\section{Herangehensweise}
Die Abschlussarbeit wird eine Literaturarbeit. In der Einführung erfolgt eine knappe Darstellung der zu behandelnden Themen Lean Startup / KMU und Anlaufmanagement. Im Hauptteil wird zunächst der Stand der Wissenschaft zum Thema Lean Startup skizziert. Den größeren Teil bildet eine umfassende Literaturanalyse zum Stand der Wissenschaft des Anlaufmanagements. Die Literaturrecherche erfolgt nach fest definierten Kriterien. Für die Literaturanalyse werden mit Hilfe des Tools \textit{Atlas.ti} alle relevanten Textstellen gecoded, d.h. identifiziert und nachvollziehbar dokumentiert. Anhand der  Ergebnisse wird anhand von möglichst vielen Quellen der Stand der Wissenschaft dargestellt. Im nächsten Abschnitt werden für das Lean Startup nicht berücksichtigte Anforderungen an das Anlaufmanagement ermittelt und daraus eine Art Anlaufmodell abgeleitet. 

Die Validierung der Ergebnisse erfolgt durch Zitierung der Quellen. Auf eine Validierung durch Experten, Fragebögen oder empirische Untersuchungen wird aufgrund des großen Umfangs verzichtet. 

\vspace{1cm}

\bibliography{../10_research/bibliography/main} % Produces the bibliography via BibTeX.

\end{document}
%
% ****** End of file apssamp.tex ******
